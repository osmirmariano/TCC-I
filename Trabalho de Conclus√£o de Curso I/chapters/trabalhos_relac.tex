% ----------------------------------------------------------------------------------------------------- %
% Capítulo 4 - TRABALHOS RELACIONADOS
% ----------------------------------------------------------------------------------------------------- %
\chapter{Trabalhos Relacionados}
\label{cap:trabalhos}

\noindent Neste capítulo está apresentados os trabalhos correlatos com ênfase em \textit{softwares} com foco no ensino e aprendizagem, que são considerados relevantes para embasamento dessa pesquisa. As seções estão estruturas em: métodos aplicados e resultados da pesquisa.

\section{Ensino e aprendizagem de cálculo: a partir do uso de \textit{softwares} matemáticos}

\noindent A presente pesquisa de Silva e Alves \cite{2016:Santana} apresenta como objetivo analisar as supostas contribuições no ensino e aprendizagem do conceito de derivadas de uma função de uma variável real com a utilização de \textit{softwares} matemáticos.

A disciplina de Cálculo Diferencial e Integral por ser de grande importância para os cursos da área de ciências exatas e por apresentar grau de dificuldade no processo do ensino e absorção por parte dos acadêmicos, alguns métodos e estratégias são aplicadas nesse processo para facilitar o ensino e aprendizagem dos acadêmicos. Diante dessa situação, ferramentas tecnológicas como \textit{softwares} são desenvolvidos com esse objetivo e consequentemente reduz os altos índices de reprovações e evasões. 

Os autores destacam um ponto importante em relação ao processo de ensino e aprendizagem, que é a metodologia aplicada, onde a metodologia utilizada por parte dos docentes ainda persiste a tradicional, a qual essa é caracterizada pelo o ensino a partir de definições, enunciados, teoremas, demostrações e exercícios, estando restrito somente ao quadro, giz e apagador. A presente metodologia faz com que a maioria dos estudantes apenas resolvam questões aplicando fórmulas, que muitas vez são memorizadas somente para resolver durante as avaliações, o que desse modo, não compreendem nenhum dos conceitos envolvidos na resolução do problema. Mas, outra metodologia tem surgido para tentar contornar essa situação que é a com utilização de recursos tecnológicos

\subsection{Métodos Aplicados}
\noindent O estudo utilizado pelos os autores abordaram a metodologia qualitativa afim de identificar \textit{softwares} matemáticos, para isso, foi realizada a seleção de trabalhos que tratavam sobre o ensino e aprendizagem de cálculo a partir do uso de \textit{softwares} matemáticos a partir desses trabalhos selecionados os mesmos foram estudados a fim de verificar os \textit{softwares} que contribuíram para o aprendizado.

\subsection{Resultados}
\noindent Com a análise realizada pelos os autores \cite{2016:Santana} concluiu que a utilização das tecnologias estão cada vez mais presentes e seu uso está crescente nos ambientes da sociedade, especialmente na educação. Constatou também que de acordo, com a escolha de bons \textit{softwares} e aplicação de uma metodologia correta para o ensino, o uso do computador e das tecnologias presentes trazem grandes vantagens para o ensino de cálculo, pois torna o aprendizado dos discentes mais proveitoso em relação a metodologia tradicional aplicada.  

\section{Criação de um \textit{software} de apoio ao ensino e à aprendizagem de Álgebra Linear}

\noindent Na dissertação de Rodrigues \cite{2009:Rodrigues} é abordada a criação de um \textit{software} para auxiliar no ensino e à aprendizagem de base e dimensão de um espaço vetorial de uma turma de licenciatura de matemática. Segundo o autor, o alto grau de abstração dos assuntos e ao grande volume de informações torna um ponto para os altos índices de reprovações e evasões dos alunos, a partir de tal situação, surge a necessidade de uma intervenção para colaborar no ensino e aprendizagem destes conteúdos. A qual, a partir das intervenções é possível identificar recursos metodológicos para ajudar no aprendizado dos acadêmicos, um exemplo, são \textit{softwares} que são desenvolvidos para amenizar esses altos índices.

\subsection{Métodos Aplicados}
\noindent O \textit{software} foi desenvolvido em três módulos, sendo eles: introdução, base e dimensão e para a avaliação do aprendizado por parte dos estudantes foram realizadas separadamente em cada módulo. Os métodos usados para verificar o aprendizado foi avaliação heurística, questionário de satisfação e avaliação de conteúdo.

A avaliação heurística foi aplicada com objetivo de identificar problemas de usabilidade do \textit{software}, o questionário de satisfação teve como objetivo verificar o grau de satisfação do usuário quanto a abordagem do conteúdo pelo \textit{software} e para a avaliação dos conteúdos o autor aplicou duas atividades escritas e individuais a fim de identificar se os alunos conseguiram abstrair o conteúdo apresentado.

\subsection{Resultados}
\noindent De acordo com a metodologia e aplicação do \textit{software} com objetivo de verificar se o sistema conseguiria melhorar o aprendizado dos alunos de uma turma de matemática em relação aos conteúdos de Álgebra Linear. Obteve-se como resultados que a partir da utilização do \textit{software} houve melhora no aprendizado dos acadêmicos em que \cite{2009:Rodrigues} destaca que os conteúdos propostos no sistema foram compreendidos pela a maioria dos estudantes participantes. Desse modo, o autor concluiu a partir da análise dos três módulos utilizados no \textit{software} que o mesmo contribuiu para o entendimento dos conceitos.

\section{O programa GAP como ferramenta de ensino e aprendizagem de Álgebra e uma reflexão das dificuldades da disciplina Álgebra I}  

Na presente pesquisa realizada por \cite{2016:Santos} são destacados pontos importantes no que tange ao aprendizado de uma turma de alunos do curso de Licenciatura de Matemática em específico da turma de Álgebra Linear I na Universidade Federal do Goiás, a qual a motivação dos autores partiu pelos os altos índices de desistência da citada disciplina. Partindo dessa situação o objetivo do estudo foi buscar novas metodologias para auxiliar no ensino e aprendizagem dos acadêmicos.

Com o advento das ferramentas tecnológicas, trouxe consigo novas formas de aprendizados e diante, os autores explanam que essas ferramentas modificam a forma como o aprendizado acontece, visto o que antes era somente o modelo tradicional aplicado para o ensino. Onde nesse modelo consiste em uma sala de aula em que o professor possui todo o conhecimento e os alunos são passíveis no ensino e os únicos recursos para o ensino presente são somente um quadro, giz e apagador. Mas com a era das ferramentas tecnológicas esses recursos foram alterados e o aprendizado dos estudantes começaram a romper as barreiras que antes estavam restritos somente a sala de aula.

\subsection{Métodos Aplicados}
\noindent A pesquisa apresentou como enfoque investigar as vantagens da utilização de tecnologias no ensino superior com o \textit{software} GAP (\textit{Groups, Algorithms, Programming - System for Computational Discrete Algebra}) para auxiliar no aprendizado. Para a análise foi aplicados questionários e entrevistas a fim de identificar pontos chaves em relação ao aprendizado dos estudantes, ao qual esses foram aplicados antes de uma aula no modelo tradicional e logo depois outra aula já com a utilização de ferramentas tecnológicas. Quanto a utilização do \textit{software} GAP, inicialmente os pesquisadores participaram de uma aula de forma tradicional e depois os mesmos participaram e prepararam de uma já com a utilização do \textit{software} GAP com objetivo de facilitar e motivar o processo de ensino e aprendizagem na disciplina de Álgebra Linear.

\subsection{Resultados}
\noindent Com a utilização do \textit{software} GAP houve despertar, curiosidades e motivação dos alunos em relação a disciplina. E diante das análises realizadas através dos questionários e entrevistas bem como nas aulas com a utilização do GAP trouxe pontos positivos no aprendizados dos acadêmicos.