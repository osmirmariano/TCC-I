% ----------------------------------------------------------------------------------------------------- %
% Capítulo 6 - CONCLUSÃO
% ----------------------------------------------------------------------------------------------------- %
\chapter{Conclusões}
\label{cap:conclusoes}

\noindent As aplicações da Álgebra Linear está presente em diversas áreas e na computação não é diferente, devido a sua aplicação na computação é necessário que o estudante de um curso dessa área tenha que aprender os conceitos e métodos da Álgebra Linear para aplicá-los em problemas, mas muitos estudantes relatam que essa área é bastante difícil. Contudo, atualmente na era tecnológica alguns recursos tecnológicos são aplicados a fim de ajudar na compreensão dos assuntos, seja na Álgebra Linear ou não.

No presente trabalho demonstra conceitos e métodos para o desenvolvimento de uma plataforma de ensino e aprendizagem, com objetivo de auxiliar os acadêmicos no aprendizado dos conteúdos da disciplina de Álgebra Linear do curso de Ciência da Computação e para verificar se o \textit{software} irá auxiliar na aprendizagem dos acadêmicos será realizada uma comparação entre duas turnas de Álgebra Linear, uma correspondente ao semestre de 2017-1 e 2017-2, ao qual a de 2017-1 os acadêmicos não irão utilizar o \textit{software} AlfaGebra e no de 2017-2 os estudante irão utilizar o sistema.

A disciplina de Álgebra Linear no curso de Ciência da Computação apresenta altos índices de evasão da disciplina, bem como também reprovações, por tal motivo é proposto nesse trabalho a inserção do \textit{software} AlfaGebra para motivar os alunos no processo de ensino e aprendizagem da disciplina de Álgebra Linear.

Portanto, espera que com a aplicação da plataforma de ensino e aprendizagem AlfaGebra possa contribuir para o aprendizado dos acadêmicos. No atual momento já foi possível identificar os aspectos referente ao aprendizado e dificuldades dos estudantes a partir da aplicação do questionário base junto a turna do período de 2017-1 para assim compreender as necessidades e pontos fortes e fracos que influência na aquisição de conhecimento.
