% ----------------------------------------------------------------------------------------------------- %
% Capítulo 1 - INTRODUÇÃO
% ----------------------------------------------------------------------------------------------------- %
\chapter{Introdução}
\label{cap:introducao}

\noindent A importância da Álgebra Linear tem crescido nas últimas décadas, principalmente nos modelos matemáticos lineares que surgem em diversas áreas, como a economia, aviação, exploração petrolífera, circuitos eletrônicos, estatística, dentre várias outras, e comumente nestes modelos aparecem a resolução de Sistemas de Equações Lineares, a qual de acordo com Leon \cite{1998:Leon} ``mais de 75\% de todos os problemas matemáticos encontrados em aplicações científicas e industriais envolvem a resolução de um sistema linear em alguma etapa" (p. 1).  

Para Ana \cite{2010:furtado}, a Álgebra Linear é vista como uma disciplina de fundamental importância para vários estudiosos, como matemáticos e cientistas que a utilizam como instrumento para resoluções de problemas. Dessarte, não é diferente que ela ``constitui uma parte importante no conteúdo matemático que é usado no início de um curso da área de exata" (\cite{1998:dirier}, apud \cite{2010:furtado}, 2010, p. 2).

Na computação gráfica, a aplicação de espaço vetorial é bastante utilizada, em que o espaço espectral de cores é um espaço vetorial de dimensão três, que são as três cores primárias, vermelho, verde e azul, este sistema é conhecido como sistema RGB (\textit{Red, Green} e \textit{Blue}).  Diferentes sistemas de coordenadas, conhecidos como sistemas de cores, são considerados neste espaço vetorial de acordo com a aplicação ou dispositivo de saída gráfica (monitor, impressora, projetor de vídeo, etc.).

Em relação ao ensino e aprendizagem da Álgebra Linear, Celestino \cite{2000:celestino} destaca a importância de pesquisas voltadas para esse estudo:

\begin{quote}
    pesquisas sobre o ensino-aprendizagem da Álgebra Linear repousa no fato de que ela hoje se encontra subjacente a quase todos os domínios da Matemática. Desta forma, é imprescindível que aqueles que pretendem trabalhar com as ciências que utilizam a Matemática, tanto como objeto de seu estudo quanto como instrumento para outros estudos, dominem seus principais conceitos. Por isso se implantou o ensino de Álgebra Linear nos diferentes cursos das chamadas Ciências Exatas, como Engenharia, Física, Química, Ciências da Computação e outras, além de Matemática (p. 9).
\end{quote}

Na Universidade Federal do Tocantins, a disciplina de Álgebra Linear faz parte da estrutura curricular dos cursos de engenharias (Civil, Elétrica, Alimentos e Biotecnológica), Ciência da Computação e Matemática. A citada disciplina é considerada por muitos alunos como difícil, o que é perceptível pelos os altos índices de reprovações e evasão dos acadêmicos. Usando como base a turma do semestre de 2015-1 de Ciência da Computação, de um total de 38 alunos somente 6 passaram com média considerada pela instituição para obter aprovação, sendo superior ou igual a 7 (sete) e muitos deles já desistem logo após a aplicação da primeira prova da disciplina.  

Desse modo, a proposta desse trabalho visa o desenvolvimento de uma plataforma de ensino e aprendizagem para a disciplina de Álgebra Linear, com objetivo de auxiliar os acadêmicos no aprendizado dos conteúdos e a partir da plataforma realizar comparação com os estudantes que cursaram a disciplina sem a utilização do \textit{software} e outra com a utilização, para que assim possa avaliar se houve melhora do aprendizado dos estudantes.

\section{Justificativa}

\noindent Nos dias atuais, na era da tecnologia, ainda existem muitos cientistas, matemáticos e engenheiros que dispõem de grande parte do seu tempo realizando pesquisas e muitos deles fazendo cálculos matemáticos manualmente. Devido a era tecnológica, muitos desses cálculos podem serem realizados com auxílios de \textit{software}, dentre alguns deles presente no mercados são: \textit{MatLab, Mathematica, Geogebra} entre outros. A partir da utilização das ferramentas de tecnologias (TICs), possibilita que as tarefas que antes era tediosas para seu desenvolvimento agora elas se tornam mais fáceis de serem desenvolvidas.

Lima \cite{2000:lima} destaca que no atual momento, os computadores se tornaram indispensáveis ao trabalho, como nas áreas da Ciência e Engenharia. E a situação nas instituições acadêmicas não é diferente, em que a cada momento as instituições estão inteiradas acerca da importância do uso de computadores como ferramenta para o ensino e aprendizagem e para isto, elas tem promovido que os acadêmicos ainda na graduação possam ter o contato direto com essas ferramentas para assim auxiliar no aprendizado.

A Álgebra Linear é apresentada como uma área que exige muito esforço no decorrer do aprendizado, visto que é considerada por muitos alunos como uma área difícil de compreender e esse assunto é tão pertinente que alguns pesquisadores já estudaram o desempenho dos acadêmicos que cursam essa disciplina na graduação. Celestino \cite{2000:celestino} em sua pesquisa de mestrado analisou os resultados de reprovações na Universidade Estadual Paulista (UNESP) e Universidade de São Paulo (USP) e identificou que as reprovações giram em torno de 25\% a 50\% e também identificou que pesquisas realizadas em outros países mostraram as dificuldades dos alunos na compreensão dos principais conceitos da Álgebra Linear.

Diante das informações supracitadas e com objetivo de auxiliar os acadêmicos no aprendizado da disciplina de Álgebra Linear do curso de Ciência da Computação, torna primordial o desenvolvimento de uma plataforma de ensino e aprendizagem que ajude a melhorar os índices de reprovações e evasão dos estudante da citada disciplina e em conjunto verificar se a plataforma irá realmente auxiliar no processo de aprendizagem, para tal, será realizada uma comparação entre duas turmas de Álgebra Linear, uma que não utilizará \textit{software} e outra que utilizará no decorrer da ministração da disciplina pelo o professor.  

\section{Objetivos}

\subsection{Objetivo Geral}

\noindent O principal objetivo deste trabalho consiste em desenvolver a plataforma de ensino e aprendizagem AlfaGebra e avaliar o aprendizado dos acadêmicos da disciplina de Álgebra Linear do curso de Ciência da Computação.

\subsection{Objetivos Específicos}

Os objetivos específicos são descritos como:

\begin{enumerate}
    \item Desenvolver os módulos de sistemas de equações lineares e espaço vetorial da plataforma AlfaGebra em versão \textit{desktop} para o sistema operacional Windows.
	
	\item Aplicar a metodologia ágil de desenvolvimento iterativo e incremental de \textit{software Scrum}.
	
	\item Comparar o nível de aprendizado dos acadêmicos da disciplina de Álgebra Linear com e sem a utilização do \textit{software} AlfaGebra.
	
	\item Aplicar o teste de usabilidade a fim de um melhor atendimento ao usuário.
\end{enumerate}

\section{Estrutura do Trabalho}

\noindent O Capítulo \ref{cap:algebra} aborda o histórico da Álgebra Linear no geral e a Álgebra Linear na Universidade Federal do Tocantins. Posteriormente, no Capítulo \ref{cap:teoria} é apresentada a fundamentação teórica utilizada no trabalho que é de grande importância para o entendimento desse trabalho. O Capítulo \ref{cap:trabalhos} apresenta os trabalhos relacionados, com destaque em aplicações utilizadas para auxiliar no ensino e aprendizagem. Já no Capítulo \ref{cap:metodologia} é apresentado todo o passo a passo do desenvolvimento desse trabalho e os métodos aplicados para o alcance dos resultados. No Capítulo \ref{cap:resultados} estão apresentados os resultados parciais e por fim, no Capítulo \ref{cap:conclusoes} são descritas as conclusões.
