
\documentclass{scrreprt}
\usepackage{listings}
\usepackage{underscore}
\usepackage[bookmarks=true]{hyperref}
\usepackage[utf8]{inputenc}
\usepackage[portuges]{babel}

\def\myversion{1.0 }
\date{}

\usepackage{hyperref}
\begin{document}

\begin{flushright}
    \rule{16cm}{5pt}\vskip1cm
    \begin{bfseries}
        \Huge{Especificação de Requisitos\\ de \textit{Software}}\\ 
        \vspace{1.9cm}
        \\
        \vspace{1.9cm}
        AlfaGebra\\
        \vspace{1.9cm}
        \LARGE{Versão \myversion Desenvolvimento}\\
        \vspace{1.9cm}
        Preparado por Osmir Custódio Mariano e Jhonatan Sousa Santiago\\
        \vspace{1.9cm}
        Universidade Federal do Tocantins\\
        \vspace{1.9cm}
        \today\\
    \end{bfseries}
\end{flushright}

\tableofcontents


\chapter*{Histórico de Revisões}

\begin{center}
    \begin{tabular}{|c|c|c|c|}
        \hline
	    Nome & Data & Descrição das alterações & Versão\\
        \hline
	    Osmir Mariano e Jhonatan Santiago & 01/08/2017 & Levantamento de Requisitos & V1.0\\
        \hline
    \end{tabular}
\end{center}

\chapter{Introdução}

\section{Objetivos deste documento}
Este documento tem por objetivo apresentar todas as informações técnicas e funcionais sobre o sistema de cálculos de Álgebra Linear, AlfaGebra. O \textit{software} visa auxiliar os acadêmicos dos cursos de Ciência da Computação, Matemática e Engenharias, mas inicialmente a proposta é somente para o curso de Ciência da Computação, depois será expandido para os demais cursos. Ao longo deste documento serão apresentadas informações pertinentes sobre a concepção do projeto, definições dos requisitos funcionais e não funcionais e interface homem-computador.

\section{Público Alvo}
O principal público alvo do sistema serão os acadêmicos dos curso de Ciência da Computação em específicos os da disciplina de Álgebra Linear. Ao qual será um sistema que apresentará recursos e conteúdo da área para que venha auxiliar no aprendizado dos mesmos.

\section{Escopo do produto}
O AlfaGebra é um \textit{software} matemático para resolução de problemas de Álgebra Linear, apresenta como objetivo auxiliar no aprendizado dos acadêmicos. O sistema por ser para o ensino e aprendizagem apresentará como requisitos uma parte voltada para os aspectos teóricos, exercícios resolvidos e opção para que o usuários forneça expressões e assim o sistema irá descrever todo o passo a passo da resolução do problema.

\subsection{Nome do produto e de seus componentes principais}
\begin{itemize}
    \item \textbf{Produto: }AlfaGebra.
    \item \textbf{Componentes principais: }Apresentação de conteúdos teóricos, exercícios resolvidos e opção para inserção de expressões para que assim o sistema possa realizar o cálculo.
\end{itemize}

\section{Descrição do produto}
O \textit{software} AlfaGebra, será desenvolvido com as seguintes características: métodos de ensino através de conteúdos teóricos, exercícios resolvidos e uma opção para inserção de expressões e assim o sistema tratar e mostrar o resultado com descrição de como chegou no resultado. 

\section{Referências}
Pressman, Roger S. Engenharia de software: uma abordagem profissional. 7ª Edição." Ed: McGraw Hill (2011).\newline

\noindent SOMMERVILLE, I.Engenharia de Software. 8. ed. [S.l.]:  Pearson Addison, 2007.

\chapter{Descrição Geral}
AlfaGebra apresenta como objetivo auxiliar os alunos da Universidade Federal do Tocantins em específico o curso de Ciência da Computação na disciplina de Álgebra Linear. Inicialmente, no aprendizado, em virtude dessa disciplina apresentar índices de evasões e reprovações elevadas. A Universidade Federal do Tocantins atualmente conta com aproximadamente de 280 (duzentos e oitenta) alunos por semestre que cursam a disciplina de Álgebra Linear nos cursos da área de exata. Com o objetivo de minimizar esses índices, o sistema propõe disponibilizar um \textit{software} em versão para \textit{desktop} para o sistema operacional Windows.

\section{Perspectiva do Produto}
Espera que com a utilização do AlfaGebra possa melhorá os altos índices de evasões da disciplina em cada semestre e também contribuir para o ensino e aprendizagem dos acadêmicos, tornando as aulas mais dinâmicas, ao invés do modelo tradicional de quadro, giz e professor.

\section{Função do Produto}
O AlfaGebra deve auxiliar os acadêmicos das áreas de exatas no aprendizado dos conteúdos da disciplina de álgebra linear: sistema de equações lineares, espaço vetorial e transformações lineares, deste modo, tentando minimizar os índices de reprovações e evasões da disciplina.

\section{Características de Classes de Usuário}
A principal características dos usuários são os acadêmicos que apresentam dificuldades em absorver os conteúdos da disciplina de Álgebra Linear e que se sentem desmotivados ao estudar com abordagem da metodologia tradicional.

\section{Ambiente Operacional}
Para o desenvolvimento da plataforma será utilizada como arquitetura de desenvolvimento corporativa a Plataforma Java Standard Edition (Java SE), juntamente com a biblioteca JavaFX para a criação de interfaces gráficas agradáveis e o ambiente de desenvolvimento integrado (IDE) será adotada o Netbeans para a codificação. E para o versionamento de código será adotada o Git e para armazenamento do repositório será utilizado a plataforma Bitbucket.

\chapter{Requisitos específicos}

\section{Identificação dos requisitos}

Por convenção e para facilitar a identificação dos requisitos, a referência é feita de acordo com o esquema abaixo:
\begin{center}
    \textbf{Identificador = [Siglas da Subseção primeira letra $\mid$  Numeração em ordem crescente]}
\end{center}

\section{Prioridades dos requisitos}
Para estabelecer a prioridade dos requisitos, foram adotadas as denominações: essencial, importante e desejável. Na tabela 1 segue a descrição do significado de cada uma dessas denominações.

\begin{table}[!htb]
    \centering
    \caption{Descrição das prioridades dos requisitos}
    \label{descricao_prioridades_requisitos}
    \begin{tabular}{|c|l|}
    \hline
    \textbf{Prioridade} & \multicolumn{1}{c|}{\textbf{Descrição}}       \\ \hline
Essencial & \begin{tabular}[c]{@{}l@{}}Estes são requisitos sem o qual o sistema não entra em funcionamento. \\ Esses requisitos são essenciais e tem que, ser implementados \\ impreterivelmente.\end{tabular}              \\ \hline
Importante & \begin{tabular}[c]{@{}l@{}}Estes são requisitos sem o qual o sistema entra em funcionamento, \\ mas de forma não satisfatória. Esses requisitos devem ser implementados, \\ mas se  não forem, poderá ser utilizado.\end{tabular} 
    \\ \hline
Desejável & \begin{tabular}[c]{@{}l@{}}Estes são os requisitos que não compromete as funcionalidades básicas\\  do sistema, ou seja, o sistema pode funcionar de forma satisfatória\\ sem ele.\end{tabular}                 \\ \hline
    \end{tabular}
\end{table}

\section{Descrição dos requisitos}
\subsection{Requisitos Funcionais}
\subsubsection{[RF01] Versão Desktop}
O sistema deve ser desenvolvido em versão para Desktop. Sendo composto por três módulos: Sistemas de Equações Lineares; Espaço Vetorial; e Transformações Lineares.
\subsubsection{Prioridade}
Essencial

\subsubsection{[RF02] Cálculo matriz linha reduzida}
O sistema deve permitir que o usuário entre com expressões matemáticas para realizar o cálculo da matriz linha reduzida à forma escada. Usando para resolução das três operações elementares. Permutação da i-ésima e j-ésima linhas (Li $\rightarrow$ Lj); multiplicação da i-ésima linha por um escalar não nulo k (Li $\rightarrow$ k*Lj); e substituição de i-ésima linha pela i-ésima linha mais k vezes a j-ésima linha (Li \rightarrow Lj + K*Lj).
\subsubsection{Prioridade}
Essencial

\subsubsection{[RF03] Demonstrar se satisfaz ou não a forma escada}
O sistema deve mostrar para o usuário, quando não satisfeito a matriz linha reduzida à forma escada, as operações que não abrangem, bem como também, mostrar que é forma escada e seus respectivos resultados.
\subsubsection{Prioridade}
Importante

\subsubsection{[RF04] Cálculo método de Gauss}
O sistema deve ser capaz de realizar cálculos utilizando o método de Gauss.
\subsubsection{Prioridade}
Essencial

\subsubsection{[RF05] Entrada de expressões}
O sistema deve possuir um campo para entrada de expressões para que o usuário entre e assim o sistema realize o cálculo.
\subsubsection{Prioridade}
Essencial

\subsubsection{[RF06] Classificação de sistema}
O sistema deve ser capaz de classificar um sistema em: uma única solução; qualquer número real terá solução; e não existe solução.
\subsubsection{Prioridade}
Importante

\subsubsection{[RF07] Apresentação gráfica}
O sistema para cálculos de classificação de sistema, deve mostrar além do resultado um gráfico de tal situação do cálculo.
\subsubsection{Prioridade}
Importante

\subsubsection{[RF08] Apresentação passo a passo}
O sistema deve mostrar o resultado e todo o passo a passo que gerou o resultado.
\subsubsection{Prioridade}
Importante

\subsubsection{[RF09] Identificação de espaço vetorial}
O sistema deverá identificar a partir de expressões fornecidas pelo o usuário se é ou não espaço vetorial.
\subsubsection{Prioridade}
Importante

\subsubsection{[RF10] Identificação de subespaço vetorial}
O sistema deverá identificar a partir de expressões fornecidas pelo o usuário se é ou não subespaço vetorial.
\subsubsection{Prioridade}
Importante

\subsubsection{[RF11] Combinação linear}
Através do sistema deverá ser possível identificar se um vetor é combinação linear de outros vetor.
\subsubsection{Prioridade}
Importante

\subsubsection{[RF12] Determinação de subespaço gerado}
O sistema deverá realizar o cálculo para determinar o subespaço gerado de um determinado vetor.
\subsubsection{Prioridade}
Importante

\subsubsection{[RF13] Dependência Linear}
O sistema deverá verificar e classificar se um determinado conjunto são linearmente dependente e linearmente independente.
\subsubsection{Prioridade}
Importante

\subsubsection{[RF14] Base de um espaço vetorial}
No sistema deverá ser possível realizar o cálculo de base de um espaço vetorial.
\subsubsection{Prioridade}
Importante

\subsubsection{[RF15] Cálculo da matriz de mudança de base}
A partir do sistema deverá ser possível realizar o cálculo da matriz de mudança de base.
\subsubsection{Prioridade}
Importante

\subsubsection{[RF16] Matriz de uma transformação linear }
O sistema deve ser capaz de representar a matriz de uma transformação linear
\subsubsection{Prioridade}
Essencial

\subsubsection{[RF17] Núcleo de uma transformação linear }
O sistema deve ser capaz de representar o núcleo de uma transformação linear
\subsubsection{Prioridade}
Essencial

\subsubsection{[RF18] Imagem de uma transformação linear }
O sistema deve ser capaz de representar a imagem de uma transformação linear
\subsubsection{Prioridade}
Essencial

\subsubsection{[RF19]  Operações com transformações lineares}
O sistema deve fazer cálculos com operações de transformações lineares, operações do tipo adição, multiplicação por escalar, etc.
\subsubsection{Prioridade}
Essencial

\subsubsection{[RF20] Calcular Dilatações ou Contrações }
O sistema deve realizar dilatações ou contrações em transformações lineares
\subsubsection{Prioridade}
Essencial

\subsubsection{[RF21] Calcular Reflexões}
O sistema deve ser capaz de realizar operações de reflexões
\subsubsection{Prioridade}
Essencial

\subsubsection{[RF22] Calcular Rotações }
O sistema deve ser capaz de realizar operações de rotações
\subsubsection{Prioridade}
Essencial

\subsubsection{[RF23] Calcular Cisalhamentos }
O sistema deve realizar operações cisalhamentos
\subsubsection{Prioridade}
Essencial


\subsection{Requisitos não funcionais}

\subsubsection{Usabilidade}
[RNF24] O sistema será desenvolvido para que o usuário utilize com facilidade e praticidade, através de uma interface agradável, textos bem visíveis e uma fácil navegação através de abas para separar e organizar as sessões (módulos). 

\subsubsection{Desempenho}
[RNF25] O sistema apresentará um tempo limite para processamento dos cálculos. Para tal é necessário que a máquina do usuário tenha pelo menos as configurações de 2GB de memória RAM, 1 processador de dois núcleos. Sendo que de acordo com as configurações que vai ser estabelecido o tempo de processamento dos cálculos.

\subsubsection{Disponibilidade}
[RNF26] O sistema estará a todo tempo disponível para o usuário, desde que o mesmo o tenha instalado em seu ambiente de estudo (Computador).

\subsubsection{Acessibilidade}
[RNF27] A interface do sistema com o usuário final deve ser adequada a adaptações e personalizações que permitam sua utilização por usuários com necessidades especiais. Essas opções devem ser compatíveis com software especializados que possam vir a ser acoplado, bem como seguir orientações específicas de acessibilidade de interface.

\subsubsection{Compatibilidade}
[RNF28] O sistema, por se tratar de um software para desktop, deverá rodar nos sistemas operacionais Windows e Linux. Para o Linux a variante utilizada é o Ubuntu a partir da versão 14.10, para o Windows utiliza-se versões a partir do XP (XP, Vista, 7, 8 e 10). O comportamento deve ser o mesmo, tanto no que se refere às funcionalidades quanto à instalação.

\subsection{Requisitos de licença}
[RL29] O sistema de ensino e aprendizagem em álgebra linear deverá ser distribuído sob a licença GNU \textit{General Public License} (Licença Pública Geral), devendo ser asseguradas às liberdades de uso, acesso ao código fonte e distribuição.

\end{document}
