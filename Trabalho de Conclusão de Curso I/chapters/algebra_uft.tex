% ----------------------------------------------------------------------------------------------------- %
% Capítulo 2 - ÁLGEBRA LINEAR NA UFT
% ----------------------------------------------------------------------------------------------------- %
\chapter{Ensino da Álgebra Linear}
\label{cap:algebra}

\noindent No decorrer desse capítulo é apresentado um breve histórico sobre a Álgebra Linear e sobre o ensino da Álgebra Linear da Universidade Federal do Tocantins e em especial no curso de Ciência da Computação.

\section{Histórico da Álgebra Linear}

\noindent A Álgebra Linear começou a dar seus primeiros passos no final do século XIX. Com o emergir dos números complexos surgiu a necessidade de relacionar a Álgebra Linear com a Geometria o que nesse contexto, Rodrigues \cite{2009:Rodrigues} explana porque essa necessidade era válida, e isso era devido a aceitação que os números complexos possibilitou na representação geométrica e pelos os testes de representá-los em três dimensões. A partir disso, motivou Willian Rowan Hamilton a constatar os quatérnios, que com eles possibilitou a criação de um sistema, em que a operação de multiplicação não dispunha das propriedades comutativas. E foi nesse ponto que os primeiros passos da Álgebra Linear começaram rumo à sua criação.

Com o surgimento da Álgebra Linear, alguns trabalhos foram desenvolvidos a fim de demonstrar esse novo ramo da matemática e dentre os trabalhos de grande destaque sobre esse estudo encontra-se nos Estados Unidos e na França com início de sua criação nos anos 80. Furtado \cite{2011:Furtado} relata que um grupo de pesquisadores na França desenvolveu um conjunto de artigos a respeito da Álgebra Linear e que logo após tornaram-se um livro, intitulado de \textit{L'Enseignement de L'Algébre Linéaire en Question} \cite{1998:dirier}, coordenado pelo o autor de maior destaque na área, Dorier. Em relação aos trabalhos desenvolvidos nos Estados Unidos, a revisão do currículo da Álgebra Linear foi realizada e então surgiu o grupo de estudo \textit{Linear Algebra Curriculum Study Group} (LACSG), sendo orientado por David Carlson. 

Dez anos depois da primeira publicação de artigos a respeito do tema Álgebra Linear, a \textit{Mathematical Association of America} (MAA) começou a formação \textit{Algebra Curriculum Study Group} (LACSG). E não demorou muito para o LACSG concluir o primeiro curso de Álgebra Linear em um curso de engenharia. No primeiro curso alguns requisitos foram aplicados, tais como a utilização de ferramentas tecnológicas para melhor aproveitamento do ensino.

O estudo da Álgebra Linear no Brasil começou a se desenvolver nos anos 90 e de acordo com Ana \cite{2010:furtado}, até 2010 no Brasil a produção de trabalhos voltados para Álgebra Linear eram poucos e a pesquisadora que mais desenvolveu pesquisas na área foi Marlene Alves Dias, que começou a trabalhar em suas produções na França. Celestino \cite{2000:celestino} desenvolveu sua dissertação de Mestrado em que estudava o histórico do ensino e aprendizagem da Álgebra Linear, ao qual focou nos poucos trabalhos que existiam no atual momento e também citou de outros países.

Alguns matemáticos realizaram grande contribuição para a evolução da Álgebra Linear entre eles Frobenius (1849-1917), Hamilton(1805-1865) Lagrange (1736-1813), Gauss, Cayley (1777-1855), entre outros \cite{2013:Teixeira}.

\section{Álgebra Linear na UFT}

\noindent A Universidade Federal do Tocantins (UFT) apresenta em seu catálogo cursos em diversas áreas, tais como: saúde, meio ambiente, tecnologia da informação, engenharia, entre outros e dentre essas áreas existem a presença dos cursos que dispõem em sua estrutura curricular disciplinas matemáticas e em especial a de Álgebra Linear, como as engenharias (Civil, Elétrica, Alimentos, e Biotecnológica), matemática e Ciência da Computação.

Na maioria dos cursos a disciplina de Álgebra Linear é ofertada no terceiro período e semestralmente são atendidos aproximadamente 280 (duzentos e oitenta) alunos em todos os \textit{campi}. Em relação a estrutura curricular em cada curso apresenta uma pequena variação, mas todos dispõem dos conteúdos de sistemas de equações lineares, espaço vetorial e transformações lineares.

\subsection{Álgebra Linear no curso de Ciência da Computação}

\noindent O curso de Ciência da Computação na Universidade Federal do Tocantins, apresenta em sua estrutura curricular a disciplina de Álgebra Linear, a qual é ofertada no terceiro período e apresenta como objetivo introduzir os conceitos fundamentais da Álgebra Linear e assim os acadêmicos possam adquirir conhecimentos e aplicá-los para resoluções de problemas ligados a computação, por exemplo, à computação gráfica que aborda muitos dos conceitos e implementação utilizadas na Álgebra Linear.

A disciplina se desenvolve em um semestre, com carga horária de 60h, sem a exigência de pré-requisitos e apresenta como conteúdos principais sistemas de equações lineares, espaço vetorial e transformações lineares.